% Options for packages loaded elsewhere
\PassOptionsToPackage{unicode}{hyperref}
\PassOptionsToPackage{hyphens}{url}
%
\documentclass[
]{article}
\usepackage{amsmath,amssymb}
\usepackage{lmodern}
\usepackage{iftex}
\ifPDFTeX
  \usepackage[T1]{fontenc}
  \usepackage[utf8]{inputenc}
  \usepackage{textcomp} % provide euro and other symbols
\else % if luatex or xetex
  \usepackage{unicode-math}
  \defaultfontfeatures{Scale=MatchLowercase}
  \defaultfontfeatures[\rmfamily]{Ligatures=TeX,Scale=1}
\fi
% Use upquote if available, for straight quotes in verbatim environments
\IfFileExists{upquote.sty}{\usepackage{upquote}}{}
\IfFileExists{microtype.sty}{% use microtype if available
  \usepackage[]{microtype}
  \UseMicrotypeSet[protrusion]{basicmath} % disable protrusion for tt fonts
}{}
\makeatletter
\@ifundefined{KOMAClassName}{% if non-KOMA class
  \IfFileExists{parskip.sty}{%
    \usepackage{parskip}
  }{% else
    \setlength{\parindent}{0pt}
    \setlength{\parskip}{6pt plus 2pt minus 1pt}}
}{% if KOMA class
  \KOMAoptions{parskip=half}}
\makeatother
\usepackage{xcolor}
\usepackage[margin=1in]{geometry}
\usepackage{color}
\usepackage{fancyvrb}
\newcommand{\VerbBar}{|}
\newcommand{\VERB}{\Verb[commandchars=\\\{\}]}
\DefineVerbatimEnvironment{Highlighting}{Verbatim}{commandchars=\\\{\}}
% Add ',fontsize=\small' for more characters per line
\usepackage{framed}
\definecolor{shadecolor}{RGB}{248,248,248}
\newenvironment{Shaded}{\begin{snugshade}}{\end{snugshade}}
\newcommand{\AlertTok}[1]{\textcolor[rgb]{0.94,0.16,0.16}{#1}}
\newcommand{\AnnotationTok}[1]{\textcolor[rgb]{0.56,0.35,0.01}{\textbf{\textit{#1}}}}
\newcommand{\AttributeTok}[1]{\textcolor[rgb]{0.77,0.63,0.00}{#1}}
\newcommand{\BaseNTok}[1]{\textcolor[rgb]{0.00,0.00,0.81}{#1}}
\newcommand{\BuiltInTok}[1]{#1}
\newcommand{\CharTok}[1]{\textcolor[rgb]{0.31,0.60,0.02}{#1}}
\newcommand{\CommentTok}[1]{\textcolor[rgb]{0.56,0.35,0.01}{\textit{#1}}}
\newcommand{\CommentVarTok}[1]{\textcolor[rgb]{0.56,0.35,0.01}{\textbf{\textit{#1}}}}
\newcommand{\ConstantTok}[1]{\textcolor[rgb]{0.00,0.00,0.00}{#1}}
\newcommand{\ControlFlowTok}[1]{\textcolor[rgb]{0.13,0.29,0.53}{\textbf{#1}}}
\newcommand{\DataTypeTok}[1]{\textcolor[rgb]{0.13,0.29,0.53}{#1}}
\newcommand{\DecValTok}[1]{\textcolor[rgb]{0.00,0.00,0.81}{#1}}
\newcommand{\DocumentationTok}[1]{\textcolor[rgb]{0.56,0.35,0.01}{\textbf{\textit{#1}}}}
\newcommand{\ErrorTok}[1]{\textcolor[rgb]{0.64,0.00,0.00}{\textbf{#1}}}
\newcommand{\ExtensionTok}[1]{#1}
\newcommand{\FloatTok}[1]{\textcolor[rgb]{0.00,0.00,0.81}{#1}}
\newcommand{\FunctionTok}[1]{\textcolor[rgb]{0.00,0.00,0.00}{#1}}
\newcommand{\ImportTok}[1]{#1}
\newcommand{\InformationTok}[1]{\textcolor[rgb]{0.56,0.35,0.01}{\textbf{\textit{#1}}}}
\newcommand{\KeywordTok}[1]{\textcolor[rgb]{0.13,0.29,0.53}{\textbf{#1}}}
\newcommand{\NormalTok}[1]{#1}
\newcommand{\OperatorTok}[1]{\textcolor[rgb]{0.81,0.36,0.00}{\textbf{#1}}}
\newcommand{\OtherTok}[1]{\textcolor[rgb]{0.56,0.35,0.01}{#1}}
\newcommand{\PreprocessorTok}[1]{\textcolor[rgb]{0.56,0.35,0.01}{\textit{#1}}}
\newcommand{\RegionMarkerTok}[1]{#1}
\newcommand{\SpecialCharTok}[1]{\textcolor[rgb]{0.00,0.00,0.00}{#1}}
\newcommand{\SpecialStringTok}[1]{\textcolor[rgb]{0.31,0.60,0.02}{#1}}
\newcommand{\StringTok}[1]{\textcolor[rgb]{0.31,0.60,0.02}{#1}}
\newcommand{\VariableTok}[1]{\textcolor[rgb]{0.00,0.00,0.00}{#1}}
\newcommand{\VerbatimStringTok}[1]{\textcolor[rgb]{0.31,0.60,0.02}{#1}}
\newcommand{\WarningTok}[1]{\textcolor[rgb]{0.56,0.35,0.01}{\textbf{\textit{#1}}}}
\usepackage{graphicx}
\makeatletter
\def\maxwidth{\ifdim\Gin@nat@width>\linewidth\linewidth\else\Gin@nat@width\fi}
\def\maxheight{\ifdim\Gin@nat@height>\textheight\textheight\else\Gin@nat@height\fi}
\makeatother
% Scale images if necessary, so that they will not overflow the page
% margins by default, and it is still possible to overwrite the defaults
% using explicit options in \includegraphics[width, height, ...]{}
\setkeys{Gin}{width=\maxwidth,height=\maxheight,keepaspectratio}
% Set default figure placement to htbp
\makeatletter
\def\fps@figure{htbp}
\makeatother
\setlength{\emergencystretch}{3em} % prevent overfull lines
\providecommand{\tightlist}{%
  \setlength{\itemsep}{0pt}\setlength{\parskip}{0pt}}
\setcounter{secnumdepth}{-\maxdimen} % remove section numbering
\ifLuaTeX
  \usepackage{selnolig}  % disable illegal ligatures
\fi
\IfFileExists{bookmark.sty}{\usepackage{bookmark}}{\usepackage{hyperref}}
\IfFileExists{xurl.sty}{\usepackage{xurl}}{} % add URL line breaks if available
\urlstyle{same} % disable monospaced font for URLs
\hypersetup{
  pdftitle={Lab 4: Heart of the (Tiny) Tiger},
  hidelinks,
  pdfcreator={LaTeX via pandoc}}

\title{Lab 4: Heart of the (Tiny) Tiger}
\author{}
\date{\vspace{-2.5em}2022/10/17}

\begin{document}
\maketitle

\emph{Agenda}: Writing functions to automate repetitive tasks; fitting
statistical models.

The \textbf{\emph{gamma}} distributions are a family of probability
distributions defined by the density functions,

\[ f(x) = \frac{x^{a-1} e^{-x/s}}{s^a \Gamma(a)} \]

where the \textbf{\emph{gamma function}}
\(\Gamma(a) = \int_{0}^{\infty}{u^{a-1} e^{-u} du}\) is chosen so that
the total probability of all non-negative \(x\) is 1. The parameter
\(a\) is called the \textbf{\emph{shape}}, and \(s\) is the
\textbf{\emph{scale}}. When \(a=1\), this becomes the exponential
distributions we saw in the first lab. The gamma probability density
function is called \texttt{dgamma()} in R. You can prove (as a calculus
exercise) that the expectation value of this distribution is \(as\), and
the variance \(as^2\). If the mean and variance are known, \(\mu\) and
\(\sigma^2\), then we can solve for the parameters,

\[ a = \frac{a^2s^2}{as^2} = \frac{\mu^2}{\sigma^2} \]
\[ s = \frac{as^2}{as} = \frac{\sigma^2}{\mu} \]

In this lab, you will fit a gamma distribution to data, and estimate the
uncertainty in the fit.

Our data today are measurements of the weight of the hearts of 144 cats.

\hypertarget{part-i}{%
\section{Part I}\label{part-i}}

\begin{enumerate}
\def\labelenumi{\arabic{enumi}.}
\tightlist
\item
  The data is contained in a data frame called \texttt{cats}, in the R
  package \texttt{MASS}. (This package is part of the standard R
  installation.) This records the sex of each cat, its weight in
  kilograms, and the weight of its heart in grams. Load the data as
  follows:
\end{enumerate}

\begin{Shaded}
\begin{Highlighting}[]
\FunctionTok{library}\NormalTok{(MASS)}
\end{Highlighting}
\end{Shaded}

\begin{verbatim}
## 
## 载入程辑包:'MASS'
\end{verbatim}

\begin{verbatim}
## The following object is masked from 'package:dplyr':
## 
##     select
\end{verbatim}

\begin{Shaded}
\begin{Highlighting}[]
\FunctionTok{data}\NormalTok{(cats)}
\FunctionTok{summary}\NormalTok{(cats)}
\end{Highlighting}
\end{Shaded}

\begin{verbatim}
##  Sex         Bwt             Hwt       
##  F:47   Min.   :2.000   Min.   : 6.30  
##  M:97   1st Qu.:2.300   1st Qu.: 8.95  
##         Median :2.700   Median :10.10  
##         Mean   :2.724   Mean   :10.63  
##         3rd Qu.:3.025   3rd Qu.:12.12  
##         Max.   :3.900   Max.   :20.50
\end{verbatim}

Run \texttt{summary(cats)} and explain the results.

\begin{enumerate}
\def\labelenumi{\arabic{enumi}.}
\setcounter{enumi}{1}
\tightlist
\item
  Plot a histogram of these weights using the \texttt{probability=TRUE}
  option. Add a vertical line with your calculated mean using
  \texttt{abline(v=yourmeanvaluehere)}. Does this calculated mean look
  correct?
\end{enumerate}

\begin{Shaded}
\begin{Highlighting}[]
\FunctionTok{par}\NormalTok{(}\AttributeTok{mfrow =} \FunctionTok{c}\NormalTok{(}\DecValTok{1}\NormalTok{, }\DecValTok{2}\NormalTok{))}
\FunctionTok{hist}\NormalTok{(cats}\SpecialCharTok{$}\NormalTok{Bwt, }\AttributeTok{probability=}\ConstantTok{TRUE}\NormalTok{); }\FunctionTok{abline}\NormalTok{(}\AttributeTok{v =} \FloatTok{2.724}\NormalTok{)}
\FunctionTok{hist}\NormalTok{(cats}\SpecialCharTok{$}\NormalTok{Hwt, }\AttributeTok{probability=}\ConstantTok{TRUE}\NormalTok{); }\FunctionTok{abline}\NormalTok{(}\AttributeTok{v =} \FloatTok{10.63}\NormalTok{)}
\end{Highlighting}
\end{Shaded}

\includegraphics{Lab4_files/figure-latex/unnamed-chunk-2-1.pdf}

\begin{Shaded}
\begin{Highlighting}[]
\FunctionTok{par}\NormalTok{(}\AttributeTok{mfrow =} \FunctionTok{c}\NormalTok{(}\DecValTok{1}\NormalTok{,}\DecValTok{1}\NormalTok{))}
\end{Highlighting}
\end{Shaded}

\begin{enumerate}
\def\labelenumi{\arabic{enumi}.}
\setcounter{enumi}{2}
\tightlist
\item
  Define two variables, \texttt{fake.mean\ \textless{}-\ 10} and
  \texttt{fake.var\ \textless{}-\ 8}. Write an expression for \(a\)
  using these placeholder values. Does it equal what you expected given
  the solutions above? Once it does, write another such expression for
  \(s\) and confirm.
\end{enumerate}

\begin{Shaded}
\begin{Highlighting}[]
\NormalTok{fake.mean }\OtherTok{\textless{}{-}} \DecValTok{10}
\NormalTok{fake.var }\OtherTok{\textless{}{-}} \DecValTok{8}
\NormalTok{a }\OtherTok{\textless{}{-}}\NormalTok{ fake.mean }\SpecialCharTok{\^{}} \DecValTok{2} \SpecialCharTok{/}\NormalTok{ fake.var}
\NormalTok{s }\OtherTok{\textless{}{-}}\NormalTok{ fake.var }\SpecialCharTok{/}\NormalTok{ fake.mean}
\end{Highlighting}
\end{Shaded}

\begin{enumerate}
\def\labelenumi{\arabic{enumi}.}
\setcounter{enumi}{3}
\tightlist
\item
  Calculate the mean, standard deviation, and variance of the heart
  weights using R's existing functions for these tasks. Plug the mean
  and variance of the cats' hearts into your formulas from the previous
  question and get estimates of \(a\) and \(s\). What are they? Do not
  report them to more significant digits than is reasonable.
\end{enumerate}

\begin{Shaded}
\begin{Highlighting}[]
\NormalTok{Hwt.mean }\OtherTok{\textless{}{-}} \FunctionTok{mean}\NormalTok{(cats}\SpecialCharTok{$}\NormalTok{Hwt)}
\NormalTok{Hwt.var }\OtherTok{\textless{}{-}} \FunctionTok{var}\NormalTok{(cats}\SpecialCharTok{$}\NormalTok{Hwt)}
\NormalTok{a }\OtherTok{\textless{}{-}}\NormalTok{ Hwt.mean }\SpecialCharTok{\^{}} \DecValTok{2} \SpecialCharTok{/}\NormalTok{ Hwt.var}
\NormalTok{s }\OtherTok{\textless{}{-}}\NormalTok{ Hwt.var }\SpecialCharTok{/}\NormalTok{ Hwt.mean}
\end{Highlighting}
\end{Shaded}

\begin{enumerate}
\def\labelenumi{\arabic{enumi}.}
\setcounter{enumi}{4}
\tightlist
\item
  Write a function, \texttt{cat.stats()}, which takes as input a vector
  of numbers and returns the mean and variances of these cat hearts.
  (You can use the existing mean and variance functions within this
  function.) Confirm that you are returning the values from above.
\end{enumerate}

\hypertarget{part-ii}{%
\section{Part II}\label{part-ii}}

\begin{enumerate}
\def\labelenumi{\arabic{enumi}.}
\setcounter{enumi}{5}
\item
  Now, use your existing function as a template for a new function,
  \texttt{gamma.cat()}, that calculates the mean and variances and
  returns the estimate of \(a\) and \(s\). What estimates does it give
  on the cats' hearts weight? Should it agree with your previous
  calculation?
\item
  Estimate the \(a\) and \(s\) separately for all the male cats and all
  the female cats, using \texttt{gamma.cat()}. Give the commands you
  used and the results.
\item
  Now, produce a histogram for the female cats. On top of this, add the
  shape of the gamma PDF using \texttt{curve()} with its first argument
  as \texttt{dgamma()}, the known PDF for the Gamma distribution. Is
  this distribution consistent with the empirical probability density of
  the histogram?
\item
  Repeat the previous step for male cats. How do the distributions
  compare?
\end{enumerate}

\end{document}
